\section{Future Work}
\label{sec:conclusion}
Tiptop is a useful application, and our extensions further enhance its utility.
However, aside from the unresolved bug, there are four major points that require further investigation.

First, our solution to the ``Too many open file descriptors bug" is unfortunately not comprehensive. On systems that have more than 819 processes and a hard file descriptor limit of 4096, this will still be an issue.
We do not believe that it is reasonable to require superuser permissions to solve this problem.
%We see two possible solutions to this.
Therefore, as one solution, it may be possible to develop a heuristic algorithm that only opens a file descriptor for processes of interest.
As an example the algorithm can create a hardware counter (i.e., open a file descriptor) for any process that has used the CPU in the last $N$ seconds.
Of course, this will require process monitoring that does not use hardware performance counters.
However, we believe that the overhead of such a strategy is justified, in order to minimize the probability of this bug occurring.
%Alternatively, we have not fully investigated the opportunities that PAPI may present in mitigating this problem.

Second, we found that tiptop is not well-supported in virtualized environments.
The man page briefly mentions this, but does not elaborate further.
Our testing confirms that data is often omitted or not reported correctly in VM instances.
We believe this requires further investigation to understand why this is the case.
Certainly, for testing purposes, the ability to test tiptop without requiring physical hardware would greatly increase the ability to verify that features are correctly implemented across CPU architectures.

Third, further investigation is require to understand if tiptop needs superuser access.
The PAPI library requires superuser permission to operate properly, and current bugs in the mainline version of tiptop can be resolved by running tiptop as a super user, too.
An understanding of these issues and a step towards a fully-featured non-superuser tiptop would be ideal.

Finally, our integration with PAPI opens tiptop to a much wider range of architectures and kernel versions. The tiptop application would benefit from a more robust testing framework that is able to automatically validate the correctness of performance counters.
As an example, it may be possible to create a suite low-level assembly programs that execute a deterministic number of instructions.
Then, this testing framework should be deployed and executed on a wide variety of configurations, in order to test the extensibility of tiptop.
Unfortunately, building such a testing framework was beyond the scope of this project, but would be an ideal next step to validate the statistics reported by tiptop.